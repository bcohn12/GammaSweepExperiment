\section{INTRODUCTION}

 The system used for all experiments consists of a carefully selected set of hardware, sofware, and materials.

For our tendons, we use [@victor material] strings instead of string or wire, so they are less prone to damage from wetting in cadaveric experiments, and so the wire stress extension is minimal.

 Two identical systems were designed, one to be used primarily for cadaveric experiments (wet lab) and another for exclusively non-biological specimens (dry lab).

\\subsection{FPGA Controller implementation} % (fold)
\label{sub:fpga_controller_implementation}
An FPGA is a field-programmable gate array- a small circuit which can represent functions as circuit implementations.
We opted to use FPGA in exploring these points:
\item FPGA's run in a deterministic time, and thus their latencies are several orders of magnitude faster than even an optimized CPU implementation.
\item FPGA's are within the realm of affordable devices, and 
\item Although GPUs are effective with floating point presision, we found that the level of precision in our FPGA's was satisfactory for our control requirements, with [@suraj how many floating point digits does the fpga deal with?]
\item With accurate timing being of paramount importance, the FPGA has low-variance in delivering results.


% subsection fpga_controller_implementation (end)

